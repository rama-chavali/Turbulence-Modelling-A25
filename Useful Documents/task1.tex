\documentclass[11pt,a4paper]{article}

\usepackage[margin=1in]{geometry}
\usepackage{amsmath, amssymb}
\usepackage{booktabs}
\usepackage{siunitx}
\usepackage{graphicx}

\sisetup{
  detect-all,
  per-mode=symbol,
  scientific-notation=true
}

\title{Task 1 -- Reference Flow Conditions}
\author{}
\date{}

\begin{document}
\maketitle

\section{Task 1 -- Definition of Reference Flow Conditions}

The objective of Task~1 is to determine the reference flow properties at the Pitot-static probe
location using the experimental data provided in Table~2. These reference quantities are later
used consistently to define inlet conditions and fluid properties in the CFD simulations.

\subsection{Experimental Reference Data}

Stagnation and static pressure and temperature measurements were obtained using a Pitot-static
probe located at $x=\SI{3.502}{m}$, $y=\SI{-0.608}{m}$, $z=\SI{0}{m}$ (Figure~2).
The experimental quantities available at this location are: the reference Mach number
$\mathrm{Ma}_{\mathrm{ref}}$, the stagnation temperature $T_{0,\mathrm{ref}}$,
the stagnation pressure $p_{0,\mathrm{ref}}$, the static pressure $p_{\mathrm{ref}}$,
and the ratio $u_{\mathrm{ref}}/\nu$.

Air is assumed to behave as a perfect gas, with $\gamma = 1.4$ and
$R = \SI{287.05}{J\,kg^{-1}\,K^{-1}}$.

\subsection{Reference Temperature and Velocity}

The static temperature at the reference location is obtained from the stagnation temperature
relation for a calorically perfect gas:
\begin{equation}
T_{\mathrm{ref}} =
\frac{T_{0,\mathrm{ref}}}
{1 + \frac{\gamma - 1}{2}\,\mathrm{Ma}_{\mathrm{ref}}^{2}}.
\end{equation}

The local speed of sound follows as
\begin{equation}
a_{\mathrm{ref}} = \sqrt{\gamma R T_{\mathrm{ref}}}.
\end{equation}

Using the definition of the Mach number, the reference velocity is
\begin{equation}
u_{\mathrm{ref}} = \mathrm{Ma}_{\mathrm{ref}}\, a_{\mathrm{ref}}.
\end{equation}

\subsection{Reference Density}

The reference density is computed from the ideal gas law using the local static pressure
and temperature:
\begin{equation}
\rho_{\mathrm{ref}} = \frac{p_{\mathrm{ref}}}{R\,T_{\mathrm{ref}}}.
\end{equation}

\subsection{Kinematic and Dynamic Viscosity}

The experimental ratio $u_{\mathrm{ref}}/\nu$ is provided in Table~2. The kinematic viscosity
is therefore obtained as
\begin{equation}
\nu = \frac{u_{\mathrm{ref}}}{\left(u_{\mathrm{ref}}/\nu\right)}.
\end{equation}
The corresponding dynamic viscosity is
\begin{equation}
\mu = \rho_{\mathrm{ref}}\,\nu.
\end{equation}

\subsection{Justification of Incompressible and Isothermal Assumptions}

For all experimental cases, the reference Mach number remains below $0.2$.
At such low Mach numbers, compressibility effects on the mean flow are negligible,
and the flow can be reasonably modeled as incompressible.

Furthermore, the temperature difference between stagnation and static conditions,
$T_{0,\mathrm{ref}} - T_{\mathrm{ref}}$, remains below approximately \SI{2.3}{K}
for all cases. This variation is negligible compared to the absolute temperature level
of approximately \SI{300}{K}, justifying the use of an isothermal flow model with
constant fluid properties.

\subsection{Summary of Reference Flow Properties}

Table~\ref{tab:ref_properties} summarizes the reference quantities obtained from the
experimental data and the above calculations.

\begin{table}[h!]
\centering
\caption{Summary of reference flow properties at the Pitot-static probe location.}
\label{tab:ref_properties}
\begin{tabular}{c c c c c c}
\toprule
Case &
$\mathrm{Ma}_{\mathrm{ref}}$ &
$u_{\mathrm{ref}}$ [m/s] &
$\rho_{\mathrm{ref}}$ [kg/m$^{3}$] &
$\nu$ [m$^{2}$/s] &
$\mu$ [Pa$\cdot$s] \\
\midrule
1 & 0.082 & 28.32 & 1.103 & $1.67\times10^{-5}$ & $1.84\times10^{-5}$ \\
2 & 0.136 & 47.01 & 1.091 & $1.68\times10^{-5}$ & $1.83\times10^{-5}$ \\
3 & 0.193 & 66.81 & 1.073 & $1.71\times10^{-5}$ & $1.84\times10^{-5}$ \\
\bottomrule
\end{tabular}
\end{table}

\end{document}
